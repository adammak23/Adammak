\documentclass[pdftex,a4paper,10pt]{report}
\usepackage{polski}
\usepackage[latin2]{inputenc}
\usepackage{mathtools}
\usepackage{xcolor}
\usepackage{hyperref}
\usepackage[pdftex]{graphicx}
\usepackage{color}
\usepackage{rotating}
\hypersetup{
  citebordercolor=white,
  linkbordercolor=white,
  urlbordercolor=white
}

\author{\color{red}{P.Pośpiech}}
\title{Super Tytuł}
\date{03.12.2014}

\newcommand{\HRule}{\rule{\linewidth}{0.5mm}}
\linespread{1.2}
\frenchspacing
\begin{document}

\input{./title.tex}

\tableofcontents

\chapter{LaTeX}
\label{cha;chap1}
\section{TeX}
\label{sec;sec1.1}
{\em TeX is a computer program for typesetting documents, created by Donald Knuth. It takes a suitably prepared computer file and converts it to a form which may be printed on many kinds of printers, including dot-matrix printers, laser printers and high-resolution typesetting machines.}\cite{wiki}
\section{LaTeX}
\label{sec;sec1.2}
{\sffamily LaTeX is a set of macros for TeX that aims at reducing the user's task to the sole role of writing the content, LaTeX taking care of all the formatting process. A number of well-established publishers now use TeX or LaTeX to typeset books and mathematical journals. It is also well appreciated by users caring about typography, consistent formatting, efficient collaborative writing and open formats.}\cite{wiki}


\chapter{MaTmA}
\label{cha;chap2}
\section{Tabelki}
\label{sec;sec2.1}
\begin{tabular}{| l | c | r |}
    \hline
    1 & 2 & 3 \\ \hline
    4 & 5 & 6 \\ \hline
    7 & 8 & 9 \\
    \hline
\end{tabular}

\begin{center}
  \begin{tabular}{|| l || c || r ||}
			\hline
			\hline
			1 & 2 & 3 \\ \hline
			4 & 5 & 6 \\ \hline
			7 & 8 & 9 \\
			\hline
			\hline
  \end{tabular}
\end{center}

\begin{tabular}{ l c r }
		jabłko & arbuz & banan \\
		gruszka & melon & pomarańcza \\
		śliwka & kiwi & coś \\
\end{tabular}
\section{Wzory}
\label{sec;sec2.2}
$$ \forall x \in X $$
$$ \cos^2 x * \sin^2 x = 1$$
$$ \frac{1}{2} + \frac{1}{2} = 1 $$ 
$$ \frac{n!}{k!(n-k)!} = \binom{n}{k} $$ 
$$ 3\times\frac{1}{2}=1\frac{1}{2} $$
$$ \sum_{i=1}^{10} t_i $$


\chapter{Rzeczy wypunktowane}
\label{cha;chap3}
{\ttfamily{\color{green} Lists often appear in documents, especially academic, as their purpose is often to present information in a clear and concise fashion. List structures in LaTeX are simply environments which essentially come in three flavors: itemize, enumerate and description.}}\cite{wiki}
\section{Itemize}
\label{sec;sec3.1}
{\scshape This environment is for your standard bulleted list of items.\cite{wiki} For example:}
\begin{itemize}
  \item The first item
  \item The second item
  \item The third etc \ldots
\end{itemize}
\section{Enumerate}
\label{sec;sec3.2}
{\bfseries The enumerate environment is for ordered lists, where by default, each item is numbered sequentially.\cite{wiki} For example:}
\begin{enumerate}
  \item The first item
  \item The second item
  \item The third etc \ldots
\end{enumerate}
\newpage
\section{Description}
\label{sec;sec3.3}
{\upshape The enumerate environment is for ordered lists, where by default, each item is numbered sequentially.\cite{wiki} We go two types of this one:}
\begin{itemize}
  \item
\begin{description}
  \item[First] The first item
  \item[Second] The second item
  \item[Third] The third etc \ldots
\end{description}
  \item
\begin{description}
  \item[First] \hfill \\
 \colorbox{orange}{ The first item }
  \item[Second] \hfill \\
 \colorbox{yellow}{ The second item }
  \item[Third] \hfill \\
 \colorbox{black}{{\color{white}The third etc \ldots}}
\end{description}
\end{itemize}

\chapter{Obrazki}
{\rmfamily The picture environment allows programming pictures directly in LaTeX. On the one hand, there are rather severe constraints, as the slopes of line segments as well as the radii of circles are restricted to a narrow choice of values.A picture environment is available in any LaTeX distribution, without the need of loading any external package.}
\label{cha;chap4}
\section{Matematyczne}
\label{sec;sec4.1}

\setlength{\unitlength}{0.75mm}
\begin{picture}(60,40)
\put(30,20){\vector(1,0){30}}
\put(30,20){\vector(4,1){20}}
\put(30,20){\vector(3,1){25}}
\put(30,20){\vector(2,1){30}}
\put(30,20){\vector(1,2){10}}
\thicklines
\put(30,20){\vector(-4,1){30}}
\put(30,20){\vector(-1,4){5}}
\thinlines
\put(30,20){\vector(-1,-1){5}}
\put(30,20){\vector(-1,-4){5}}
\end{picture}

\begin{center}
\setlength{\unitlength}{5cm}
\begin{picture}(1,1)
\put(0,0){\line(0,1){1}}
\put(0,0){\line(1,0){1}}
\put(0,0){\line(1,1){1}}
\put(0,0){\line(1,2){.5}}
\put(0,0){\line(1,3){.3333}}
\put(0,0){\line(1,4){.25}}
\put(0,0){\line(1,5){.2}}
\put(0,0){\line(1,6){.1667}}
\put(0,0){\line(2,1){1}}
\put(0,0){\line(2,3){.6667}}
\put(0,0){\line(2,5){.4}}
\put(0,0){\line(3,1){1}}
\put(0,0){\line(3,2){1}}
\put(0,0){\line(3,4){.75}}
\put(0,0){\line(3,5){.6}}
\put(0,0){\line(4,1){1}}
\put(0,0){\line(4,3){1}}
\put(0,0){\line(4,5){.8}}
\put(0,0){\line(5,1){1}}
\put(0,0){\line(5,2){1}}
\put(0,0){\line(5,3){1}}
\put(0,0){\line(5,4){1}}
\put(0,0){\line(5,6){.8333}}
\put(0,0){\line(6,1){1}}
\put(0,0){\line(6,5){1}}
\end{picture}
\end{center}

\section{Kurczak}
\label{sec;sec4.2}
\begin{center}
\begin{turn}{30}
\includegraphics{kurczak.png}
\end{turn}
\includegraphics[scale=0.5, angle=180]{kurczak.png}
\includegraphics[width=2.5cm]{kurczak.png}
\includegraphics{kurczak.png}
\end{center}
\begin{figure}[h!]
  \caption{Oto kurczak}
  \centering
    \includegraphics[width=0.5\textwidth]{kurczak.png}
\end{figure}


\begin{thebibliography}{1}
\pagecolor{blue}
\bibitem{wiki}\url{http://en.wikibooks.org/wiki/LaTeX}

\end{thebibliography}

\end{document}
